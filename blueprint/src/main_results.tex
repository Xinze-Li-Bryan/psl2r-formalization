\chapter{Main Results}

\section{The Main Equivalence Theorem}

The central result of this formalization is:

\begin{theorem}[Main Equivalence]
\label{thm:main_equivalence}
\lean{PSL2R.main_equivalence}
\uses{def:PSL, def:unit_tangent_bundle, def:isom_plus}
There exist natural isomorphisms:
\begin{align*}
T^1\mathbb{H} &\cong \mathrm{PSL}(2,\mathbb{R}) \\
\mathrm{PSL}(2,\mathbb{R}) &\cong \mathrm{Isom}^+(\mathbb{H}^2)
\end{align*}
These isomorphisms are compatible with the natural topological and group structures.
\end{theorem}

The proof proceeds through three main steps:

\section{PSL(2,ℝ) acts by isometries}

\begin{theorem}
\label{thm:psl2r_to_isom}
\lean{PSL2R.toIsometry}
\uses{def:PSL, def:moebius_action, def:isom_plus}
Every element of $\mathrm{PSL}(2,\mathbb{R})$ acts as an orientation-preserving isometry of $\mathbb{H}$ via Möbius transformations.
\end{theorem}

\begin{theorem}
\label{thm:psl2r_isom_bijection}
\uses{thm:psl2r_to_isom}
The map $\mathrm{PSL}(2,\mathbb{R}) \to \mathrm{Isom}^+(\mathbb{H}^2)$ is a group isomorphism.
\end{theorem}

\section{Connection to the unit tangent bundle}

\begin{theorem}
\label{thm:t1h_to_psl2r}
\lean{PSL2R.unitTangentEquiv}
\uses{def:unit_tangent_bundle, def:PSL}
There exists a natural homeomorphism
\[ T^1\mathbb{H} \cong \mathrm{PSL}(2,\mathbb{R}). \]
\end{theorem}

\begin{proof}[Proof sketch]
Given a point $(z, v) \in T^1\mathbb{H}$, we associate the unique element of $\mathrm{PSL}(2,\mathbb{R})$ that maps the base point $(i, \partial/\partial x)$ to $(z, v)$. This gives a bijection that can be shown to be continuous in both directions.
\end{proof}

\section{Future Extensions}

Potential extensions of this formalization include:
\begin{itemize}
\item Fuchsian groups and quotient surfaces
\item Teichmüller theory
\item The three-dimensional case: $\mathrm{PSL}(2,\mathbb{C})$ and hyperbolic 3-space
\end{itemize}
